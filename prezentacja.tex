\documentclass[10pt]{beamer}

% --- Pakiety ---
\usepackage[utf8]{inputenc}
\usepackage[T1]{fontenc}
\usepackage[polish]{babel}
\usepackage{tikz} % SKILL: Rysowanie diagramów
\usetikzlibrary{shapes.geometric, arrows}
\usepackage{booktabs}
\usepackage{listings}

% --- Motyw ---
\usetheme{Warsaw} % Bardziej "bogaty" temat
\usecolortheme{seahorse}

% --- Konfiguracja Stopki ---
\setbeamertemplate{footline}[frame number] % Prosta stopka z numerem
\setbeamerfont{page number in head/foot}{size=\large}

% --- Dane ---
\title{Symulator Strategii Wyścigowych F1}
\subtitle{Implementacja silnika fizycznego w C++}
\author{Jan Kowalski}
\institute{Politechnika Wrocławska\\Wydział Informatyki}
\date{\today}

% --- Definicje Diagramów (TikZ) ---
\tikzstyle{startstop} = [rectangle, rounded corners, minimum width=3cm, minimum height=1cm,text centered, draw=black, fill=red!30]
\tikzstyle{process} = [rectangle, minimum width=3cm, minimum height=1cm, text centered, draw=black, fill=orange!30]
\tikzstyle{decision} = [diamond, minimum width=3cm, minimum height=1cm, text centered, draw=black, fill=green!30]
\tikzstyle{arrow} = [thick,->,>=stealth]

\begin{document}

% 1. Tytuł
\begin{frame}
    \titlepage
\end{frame}

% 2. Agenda
\begin{frame}{Agenda}
    \tableofcontents
\end{frame}

% --- Sekcja: Cel ---
\section{Wprowadzenie}
\begin{frame}{Cel Projektu}
    Celem projektu było stworzenie aplikacji konsolowej symulującej wyścig F1 z naciskiem na:
    
    \begin{itemize}
        \item<1-> \textbf{Realizm Danych:} Statystyki kierowców z sezonu 2025.
        \item<2-> \textbf{Model Fizyczny:} Przyczepność, opory ruchu, zużycie opon.
        \item<3-> \textbf{Wydajność:} Szybka symulacja (1.5h wyścigu w 10 sekund).
    \end{itemize}
    
    \vspace{0.5cm}
    \onslide<4>{
        \begin{alertblock}{Wyzwanie}
            Najtrudniejszym elementem było wyeliminowanie błędów "kwantyzacji czasu" przy finiszu.
        \end{alertblock}
    }
\end{frame}

% --- Sekcja: Architektura ---
\section{Architektura}
\begin{frame}{Schemat Działania (Game Loop)}
    \centering
    \begin{tikzpicture}[node distance=2cm]
        \node (start) [startstop] {Start Symulacji};
        \node (update) [process, below of=start] {Aktualizacja (dt)};
        \node (check) [decision, below of=update, yshift=-0.5cm] {Koniec?};
        \node (render) [process, right of=update, xshift=3cm] {Render (Konsola)};
        \node (stop) [startstop, below of=check] {Wyniki};

        \draw [arrow] (start) -- (update);
        \draw [arrow] (update) -- (check);
        \draw [arrow] (check) -- node[anchor=east] {Tak} (stop);
        \draw [arrow] (check) -- node[anchor=south] {Nie} (render);
        \draw [arrow] (render) |- (update);
    \end{tikzpicture}
\end{frame}

\begin{frame}{Struktura Klas (C++)}
    System oparto na trzech głównych klasach:
    
    \begin{columns}
        \column{0.5\textwidth}
        \textbf{1. Race}
        \begin{itemize}
            \item Zarządza czasem globalnym.
            \item Sortuje tabelę wyników.
            \item Obsługuje wyświetlanie (ANSI).
        \end{itemize}
        
        \column{0.5\textwidth}
        \textbf{2. Car}
        \begin{itemize}
            \item Oblicza prędkość $v = f(r, \mu)$.
            \item Symuluje zużycie opon.
            \item Interpoluje czas mety.
        \end{itemize}
    \end{columns}
    
    \vspace{1cm}
    \begin{exampleblock}{Przykład kodu: Interpolacja}
        \texttt{correction = overshoot / speed;} \\
        \texttt{finishTime = raceTime - correction;}
    \end{exampleblock}
\end{frame}

% --- Sekcja: Wyniki ---
\section{Demonstracja}
\begin{frame}{Interfejs Użytkownika}
    Aplikacja działa w trybie tekstowym z odświeżaniem 20Hz.
    
    \vspace{0.5cm}
    \begin{figure}
        \includegraphics[width=0.7\textwidth]{screen_wyniki.png} % Pamiętaj o usunięciu [demo] w preambule
        \caption{Widok tabeli wyników na żywo}
    \end{figure}
\end{frame}

\begin{frame}{Analiza Danych (Sezon 2025)}
    \begin{table}
        \centering
        \small
        \begin{tabular}{l c c c} 
            \toprule
            \textbf{Zespół} & \textbf{Top Speed} & \textbf{Pit Stop} & \textbf{Przewaga} \ 
            \midrule
            \color{orange} McLaren & 96.0 & 1.05 & \textbf{Szybkość} \ 
            \color{red} Ferrari & 95.8 & 0.85 & \textbf{Strategia} \ 
            \color{blue} Red Bull & 95.5 & 0.88 & Balans \ 
            \bottomrule
        \end{tabular}
        \caption{Parametry czołowych zespołów w symulacji}
    \end{table}
\end{frame}

% --- Sekcja: Podsumowanie ---
\section{Podsumowanie}
\begin{frame}{Wnioski i Plany}
    \textbf{Osiągnięte cele:}
    \begin{itemize}
        \item [\checkmark] Płynna animacja w konsoli.
        \item [\checkmark] Poprawne odwzorowanie różnic czasowych.
        \item [\checkmark] Modularny kod (łatwy do rozwoju).
    \end{itemize}
    
    \vspace{1cm}
    \centering
    \Huge \textbf{Dziękuję za uwagę!}
\end{frame}

\end{document}